Nicolaus Copernicus (German: Nikolaus Kopernikus; Italian: Nicolo Copernico; Polish: About this sound Mikolaj Kopernik (help�info)) (19 February 1473 � 24 May 1543) was a Renaissance astronomer and the first person to formulate a comprehensive heliocentric cosmology which displaced the Earth from the center of the universe.[1]


% Miko�aj Kopernik (�ac. Nicolaus Copernicus[1], niem. Nikolaus Kopernikus; ur. 19 lutego 1473 w Toruniu, zm. 24 maja 1543 we Fromborku) � polski astronom, autor dzie�a De revolutionibus orbium coelestium[2] przedstawiaj�cego szczeg�owo i w naukowo u�ytecznej formie heliocentryczn� wizj� Wszech�wiata. Wprawdzie koncepcja heliocentryzmu pojawi�a si� ju� w staro�ytnej Grecji (jej tw�rc� by� Arystarch z Samos[3]), to jednak dopiero dzie�o Kopernika dokona�o prze�omu i wywo�a�o jedn� z najwa�niejszych rewolucji naukowych od czas�w staro�ytnych, nazywan� przewrotem kopernika�skim[4].

% Od 1497 roku sprawowa� funkcj� kanonika warmi�skiego, od 1503 scholastyka wroc�awskiego, a od 1511 kanclerza kapitu�y warmi�skiej.

% By� wybitnym polihistorem Renesansu, zajmowa� si� mi�dzy innymi astronomi�, matematyk�, prawem, ekonomi�, strategi� wojskow�, astrologi�[5][6], by� tak�e lekarzem oraz t�umaczem.
